\documentclass[oneside]{article}
\usepackage[a4paper, margin=3cm, top=4cm]{geometry}
\usepackage[ngerman]{babel}
\usepackage[parfill]{parskip}
\usepackage{graphicx}
\usepackage{fancyhdr}
\usepackage{extramarks}
\usepackage{amsmath}
\usepackage{amsthm}
\usepackage{amsfonts}
\usepackage{tikz}
\usepackage{hyperref}

% --- Hyperref Setup ---
\hypersetup{
    colorlinks=true,
    linkcolor=blue,
    filecolor=magenta,      
    urlcolor=blue,
}

% --- Page Setup ---
\linespread{1.1}

\pagestyle{fancy}
\lhead{\hmwkAuthor}
\chead{\hmwkSubject: \hmwkTitle}
\rhead{\firstxmark}
\cfoot{\thepage}

\renewcommand\headrulewidth{0.4pt}
\renewcommand\footrulewidth{0.4pt}

\setlength\parindent{0pt}

\graphicspath{ {./} }

% --- Metadata ---
\newcommand{\hmwkTitle}{Homework \#1}
\newcommand{\hmwkAuthor}{flofriday}
\newcommand{\hmwkDueDate}{10.09.2020}
\newcommand{\hmwkSubject}{Math}

\title{
    \textmd{\textbf{\hmwkSubject: \hmwkTitle}}
}
\author{\hmwkAuthor}
\date{Abgabedatum: \hmwkDueDate} 

% --- Begin Document ---
\begin{document}

\maketitle
\vspace{2cm}
\tableofcontents
\pagebreak


\section*{Problem A}
\addcontentsline{toc}{section}{Problem A}
\extramarks{Problem A}\\
Eine ausführliche Beschreibung des zu lösenden Problemes

\textbf{Lösung}\\
Eine Lösung des Problemes

\section*{Problem B}
\addcontentsline{toc}{section}{Problem B}
\extramarks{Problem B}\\
Eine ausführliche Beschreibung des zu lösenden Problemes

\textbf{Lösung}\\
Eine Lösung des Problemes

\section*{Problem C}
\addcontentsline{toc}{section}{Problem C}
\extramarks{Problem C}\\
Eine ausführliche Beschreibung des zu lösenden Problemes

\textbf{Lösung}\\
Eine Lösung des Problemes

\section*{Additional Task}
\addcontentsline{toc}{section}{Additional Task}
\extramarks{Additional Task}\\
Eine ausführliche Beschreibung des zu lösenden Problemes

\textbf{Lösung}\\
Eine Lösung des Problemes


\end{document}